\section{Propellant Feed System and Anode}

% storage tank
% feed plumbing
% injector

% talk about anode

The propellant feed system refers to the manner in which propellant is delivered to the discharge channel. The propellant would be some sort of neutral gas like Xenon, Argon and Krypton or some volatile liquid or solid propellant.


While configurations of \ac{HET} exist where the anode and the feed stream are structurally separate, in smaller form factor configurations, the anode is structurally "baked in" to the feed system, to ensure uniform flow. The advantage of uniform flow is that it prevents uneven propellant particle flux to ionization region i.e; the discharge channel, which would result in asymmetric thrust vectors, diminished propellant ionization, and decreased total efficiency.


Another benefit of having the propellant feed baked into the anode is that it allows for the uniform azimuthal flow of the propellant into the discharge chamber. An azimuthal release allows for a sustained ionization by allowing the propellant to stay in the discharge chamber for longer. Ths has resulted in improved propellant utilization thereby improving thruster efficiency.


A propellant feed system would commonly include some or all of the following components; A high-pressure storage tank for propellant storage, pressure regulators to bring the pressure down to workable pressures, a flow restrictor/orifice to set the propellant flow rate passively, mass flow controller to set the propellant flow rate actively, and stainless steel or titanium Tubing and fittings to carry the propellant between these different components. 
