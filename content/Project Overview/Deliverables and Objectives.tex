\section{Capstone Outline}
To tackle these issues this capstone project aims to explore the design, simulation, and prototyping of an electric propulsion system suitable for nanosatellites (1–10 kg). First a suitable method of electric propulsion will be selected based on mission requirements and compatibility with nanosatellite applications. The team will then design an electric propulsion module consisting of the key subsystems: a \ac{PPU}, a propellant storage and feed system, an ionization system.

The proposed design will evaluated using simulation softwares to assess performance metrics such as thrust, electrical performance, and orbital maneuvering potential. Finally, the team will develop a ground-based test stand prototype. While thruster testing may not be possible, this will lay the framework to measure and analyze critical parameters such as thrust, specific impulse, efficiency, and overall system stability under representative operating conditions.

\section{Capstone Requirements}
The primary requirements for the electric propulsion system are as follows:
\begin{itemize}
    \item The propulsion system must be compatible with nanosatellite platforms in the 1–10 kg range
    \item The propulsion module should fit within a volume of 3U (30 cm x 10 cm x 10 cm) or smaller
    \item The system should be capable of producing a thrust level sufficient for orbit keeping of a nanosatellite at 200km altitude
    \item The total power consumption of the propulsion system should not exceed 500W
    \item The system should provide a $I_{sp}$ greater than 500 seconds
    \item The propulsion system and \ac{PPU} should not exceed a total mass of 3kg
    \item The propulsion system should utilize a propellant that is safe and easy to handle
    \item The design should include considerations for thermal management to ensure stable operation in space environments
\end{itemize}

\section{Capstone Deliverables}

The expected deliverables are as follows:

\begin{itemize}
    \item A prototype electric propulsion module suitable for nanosatellites in the 1–10 kg range
    \item A design and prototype of a thrust test stand capable of accurately measuring low thrust levels
    \item CAD models, technical drawings, and electronic schematics associated with the system
    \item Simulation results validating key aspects of the propulsion system, including thrust and electrical performance
\end{itemize}

These outcomes will demonstrate the feasibility of implementing electric propulsion on nanosatellite platforms and provide a foundation for further development and testing	