\chapter{Hall Thruster Theory}
\label{chap:Hall Thruster Theory}

Before diving into the design specifics of our \ac{HET}, it is essential to understand the fundamental principles that govern its operation. In comparison to other electric propulsion systems, such as ion thrusters, the construction of a \ac{HET} may be relatively simple, but the physics involved to produce thrust is much more complicated, and less predictable. The dimensions of the discharge chamber, and the shape and strength of the magnetic field have a great impact on the performance, efficiency, and lifetime of the thruster.

% The Hall thruster (HT) 1 is an electromagnetic propulsion device that uses a cross-field plasma discharge to
% accelerate ions. The thrust is a reaction force to this acceleration, exerted upon the thruster magnetic circuit. In a
% conventional HT, axial electric and radial magnetic fields are applied in an annular channel. The magnetic field is
% large enough to lock the electrons in the azimuthal E×B drift, but small enough to leave the ion trajectories almost
% unaffected. Because of the reduced electron mobility across the magnetic field, a substantial axial electric field can
% be maintained in the quasineutral plasma and electrons can effectively ionize neutral atoms of the propellant gas.
% Under such conditions, the electric field supplies energy mainly to accelerate the unmagnetized ions. Unlike the
% space-charge limited gridded ion engine, the HT accelerates the ions in the quasineutral plasma. Thus, larger ion
% current and thrust densities can be achieved.

% Hall thrusters are relatively simple-appearing devices consisting of a cylindrical channel
% with an interior embedded anode, a magnetic circuit that generates primarily a radial
% magnetic field across the channel, and a cathode that injects electrons into the near-thruster
% region external to the channel. However, Hall thrusters rely on much more complicated
% physics compared to ion thrusters due to the E×B plasma discharge that accelerates ions
% and produces thrust. The details of the channel structure and magnetic field shape
% determine the performance, efficiency, and life of the thruster [1-5]. The efficiency and
% specific impulse (Isp) of flight-model Hall thrusters are approaching that achievable in ion
% thrusters [6, 7], but the thrust-to-power ratio and the beam current density are higher, and
% the total impulse can be comparable. Hall thrusters are simpler to build than ion thrusters
% due to the lack of precision-aligned close-space grids, and they require fewer power
% supplies and propellant-flow controllers to operate. 